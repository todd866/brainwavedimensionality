\documentclass[11pt]{letter}
\usepackage[margin=1in]{geometry}
\usepackage{hyperref}

\signature{Ian Todd\\
\texttt{itod2305@uni.sydney.edu.au}\\
Sydney Medical School, University of Sydney}

\address{Ian Todd\\
Sydney Medical School\\
University of Sydney\\
NSW 2006, Australia}

\date{\today}

\begin{document}

\begin{letter}{Editor\\
Journal of Computational Neuroscience\\
Springer}

\opening{Dear Editor,}

I am pleased to submit the manuscript titled \textbf{``The Dimensional Hierarchy of Cortical Oscillations: From Analog Substrate to Symbolic Codes''} for consideration at the \textit{Journal of Computational Neuroscience}.

\textbf{Summary.} This paper proposes that cortical frequency bands implement a cascade of information bottlenecks distinguished by spatial participation---how many oscillators engage coherently---rather than latent state-space dimensionality. The framework unifies three computational results:

\begin{enumerate}
    \item \textbf{Graph Laplacian analysis} shows that slow eigenmodes on modular networks engage substantially more oscillators than fast modes ($r = -0.75$), establishing the high-dimensional geometric substrate that slow oscillations provide.

    \item \textbf{Encoder-decoder networks} demonstrate that discrete symbolic codes emerge at a critical bottleneck width of $k \approx 2$, while $k \geq 3$ preserves continuous dynamics capable of analog manipulation.

    \item \textbf{Topological analysis} shows that self-referential structures (e.g., cyclic inference) in our framework require $k \geq 3$ to avoid trajectory collisions---a constraint that manifests as irreducible reconstruction error in linear autoencoders.
\end{enumerate}

\textbf{Experimental predictions.} Section 4.6 offers concrete, falsifiable predictions for high-density ECoG and MEG: slow-band activity should show higher participation ratios than gamma; beta should occupy an intermediate regime; and developmental or pharmacological manipulations that disrupt slow oscillations should shift the system toward premature discretization. These predictions are testable with existing recording technology.

\textbf{Fit with JCN.} The manuscript is theoretical but grounded in standard computational methods (spectral graph theory, information bottleneck, dimensionality reduction), addressing a gap between temporal and geometric notions of dimensionality in current oscillation theories. More speculative interpretations (e.g., links to development and psychopathology) are explicitly framed as hypotheses and clearly separated from the core computational results.

\textbf{Suggested reviewers.} I respectfully suggest the following experts who would be well-qualified to review this manuscript:
\begin{itemize}
    \item Earl K.\ Miller (MIT) --- working memory and prefrontal oscillations
    \item Pascal Fries (Ernst Str\"ungmann Institute) --- communication through coherence
    \item Michael Lundqvist (MIT) --- beta/gamma burst dynamics
\end{itemize}

I confirm this work is original, not under consideration elsewhere, and that I, as sole author, approve its submission.

\closing{Thank you for your consideration.\\[1em]
Sincerely,}

\end{letter}
\end{document}
